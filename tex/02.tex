\section*{Esercizio 2}

Otteniamo \texttt{((x * 2.0) * exp(y))}. Nel seguito usiamo il simbolo
``\mintinline{java}{// ...}'' per denotare che la classe contiene tutti i
metodi e i costruttori definiti nei precedenti esercizi.

\subsubsection*{Expr.java}
\inputminted{java}{tex/src/2/Expr.java}

\subsubsection*{BinaryExpr.java}
\inputminted{java}{tex/src/2/BinaryExpr.java}

\subsubsection*{AddExpr.java}
\inputminted{java}{tex/src/2/AddExpr.java}

\subsubsection*{MulExpr.java}
\inputminted{java}{tex/src/2/MulExpr.java}

\subsubsection*{SubExpr.java}
\inputminted{java}{tex/src/2/SubExpr.java}

\subsubsection*{UnaryExpr.java}
\inputminted{java}{tex/src/2/UnaryExpr.java}

\subsubsection*{NegExpr.java}
\inputminted{java}{tex/src/2/NegExpr.java}

\subsubsection*{ExpExpr.java}
\inputminted{java}{tex/src/2/ExpExpr.java}

\subsubsection*{DoubleExpr.java}
\inputminted{java}{tex/src/2/DoubleExpr.java}

\subsubsection*{ConstExpr.java}
\inputminted{java}{tex/src/2/ConstExpr.java}

