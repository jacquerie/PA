\section*{Esercizio 1}

Implementiamo un'espressione applicando l'Interpreter pattern\footnote{Gamma,
E.; Helm, R.; Johnson, R.; Vlissides, J. (1994). Design Patterns: Elements of
Reusable Object-Oriented Software. Addison-Wesley. ISBN 0-201-63361-2}, in cui
le espressioni terminali sono le classi \texttt{DoubleExpr} e
\texttt{ConstExpr}, la classe \texttt{Function} è il cliente e le restanti
classi rappresentano espressioni non terminali.

\subsubsection*{Expr.java}
\inputminted{java}{tex/src/1/Expr.java}

\subsubsection*{BinaryExpr.java}
\inputminted{java}{tex/src/1/BinaryExpr.java}

\subsubsection*{AddExpr.java}
\inputminted{java}{tex/src/1/AddExpr.java}

\subsubsection*{MulExpr.java}
\inputminted{java}{tex/src/1/MulExpr.java}

\subsubsection*{SubExpr.java}
\inputminted{java}{tex/src/1/SubExpr.java}

\subsubsection*{UnaryExpr.java}
\inputminted{java}{tex/src/1/UnaryExpr.java}

\subsubsection*{NegExpr.java}
\inputminted{java}{tex/src/1/NegExpr.java}

\subsubsection*{ExpExpr.java}
\inputminted{java}{tex/src/1/ExpExpr.java}

\subsubsection*{DoubleExpr.java}
\inputminted{java}{tex/src/1/DoubleExpr.java}

\subsubsection*{ConstExpr.java}
\inputminted{java}{tex/src/1/ConstExpr.java}

\subsubsection*{Function.java}
\inputminted{java}{tex/src/1/Function.java}
