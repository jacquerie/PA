\section*{Esercizio 4}

Otteniamo il codice ottimizzato \texttt{((1.0 * 2.0) * exp(2.5))}.  Abbiamo
usato il double dispatch del metodo \mintinline{java}{simplify} per generare la
chiamata corretta nelle classi concrete \texttt{AddExpr}, \texttt{MulExpr},
\texttt{SubExpr} e \texttt{NegExpr}. Possiamo semplificare espressioni del tipo
\texttt{x + 0}, \texttt{x * 0}, \texttt{0 - x} e \texttt{-(-x)}, ma non
\texttt{(m * n) * x}.

\subsubsection*{Expr.java}
\inputminted{java}{tex/src/4/Expr.java}

\subsubsection*{BinaryExpr.java}
\inputminted{java}{tex/src/4/BinaryExpr.java}

\subsubsection*{AddExpr.java}
\inputminted{java}{tex/src/4/AddExpr.java}

\subsubsection*{MulExpr.java}
\inputminted{java}{tex/src/4/MulExpr.java}

\subsubsection*{SubExpr.java}
\inputminted{java}{tex/src/4/SubExpr.java}

\subsubsection*{UnaryExpr.java}
\inputminted{java}{tex/src/4/UnaryExpr.java}

\subsubsection*{NegExpr.java}
\inputminted{java}{tex/src/4/NegExpr.java}

\subsubsection*{ExpExpr.java}
\inputminted{java}{tex/src/4/ExpExpr.java}

\subsubsection*{DoubleExpr.java}
\inputminted{java}{tex/src/4/DoubleExpr.java}

\subsubsection*{ConstExpr.java}
\inputminted{java}{tex/src/4/ConstExpr.java}

